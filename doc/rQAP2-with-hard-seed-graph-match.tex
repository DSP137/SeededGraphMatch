%% LyX 2.0.0 created this file.  For more info, see http://www.lyx.org/.
%% Do not edit unless you really know what you are doing.
\documentclass[english]{article}
\usepackage[T1]{fontenc}
\usepackage[latin9]{inputenc}
\usepackage{geometry}
\geometry{verbose,lmargin=1.5cm}
\usepackage{amsmath}

\makeatletter
%%%%%%%%%%%%%%%%%%%%%%%%%%%%%% User specified LaTeX commands.
\DeclareMathOperator{\Tr}{Tr}

\makeatother

\usepackage{babel}
\begin{document}
The objective function for rQAP2 is

\begin{align*}
f(P) & = & \left\Vert AP^{*}-P^{*}B\right\Vert _{F}\\
 & = & \left\Vert A_{21}-PB_{21}\right\Vert _{F} & + & \left\Vert A_{12}P-B_{12}\right\Vert _{F} & +\left\Vert A_{22}P-PB_{22}\right\Vert _{F} & \textrm{Terms (1), (2) and (3)}
\end{align*}


where $P^{*}$ is the omnibus permutation matrix $\left[\begin{array}{cc}
I & \mathbf{0}\\
\mathbf{0} & P
\end{array}\right]$ .

Consider term (1)

\begin{align*}
\left\Vert A_{21}-PB_{21}\right\Vert _{F} & = & \Tr\left[\left(A_{21}-PB_{21}\right)^{T}\left(A_{21}-PB_{21}\right)\right]\\
 & = & \Tr\left[A_{21}^{T}A_{21}-B_{21}^{T}P^{T}A_{21}-A_{21}^{T}PB_{21}+B_{21}^{T}P^{T}PB_{21}\right]\\
 & = & \Tr\left[A_{21}^{T}A_{21}-B_{21}^{T}P^{T}A_{21}-A_{21}^{T}PB_{21}+P^{T}PB_{21}B_{21}^{T}\right]\\
 & = & \Tr\left[A_{21}^{T}A_{21}-2*B_{21}^{T}P^{T}A_{21}+P^{T}PB_{21}B_{21}^{T}\right]
\end{align*}


where the simplification in the last line is due to the fact that
the matrices with minus signs in front are transposes of each other.
The three terms inside the brackets in the last line are referred
as (1.1),(1.2) and (1.3), respectively.

Similarly for term (2)

\begin{align*}
\left\Vert A_{12}P-B_{12}\right\Vert _{F} & = & \Tr\left[\left(A_{12}P-B_{12}\right)^{T}\left(A_{12}P-B_{12}\right)\right]\\
 & = & \Tr\left[P^{T}A_{12}^{T}A_{12}P-B_{12}^{T}A_{12}P-P^{T}A_{12}^{T}B_{12}+B_{12}^{T}B_{12}\right]\\
 & = & \Tr\left[PP^{T}A_{12}^{T}A_{12}-B_{12}^{T}A_{12}P-P^{T}A_{12}^{T}B_{12}+B_{12}^{T}B_{12}\right]\\
 &  & \Tr\left[PP^{T}A_{12}^{T}A_{12}-2P^{T}A_{12}^{T}B_{12}+B_{12}^{T}B_{12}\right]
\end{align*}


The three terms inside the brackets are referred as (2.1),(2.2) and
(2.3), respectively.

and finally term (3)

\begin{align*}
\left\Vert A_{22}P-PB_{22}\right\Vert _{F} & = & \Tr\left[\left(A_{22}P-PB_{22}\right)^{T}\left(A_{22}P-PB_{22}\right)\right]\\
 & = & \Tr\left[P^{T}A_{22}^{T}A_{22}P-B_{22}^{T}P^{T}A_{22}P-P^{T}A_{22}^{T}PB_{22}+B_{22}^{T}P^{T}PB_{22}\right]\\
 & = & \Tr\left[PP^{T}A_{22}^{T}A_{22}-B_{22}^{T}P^{T}A_{22}P-P^{T}A_{22}^{T}PB_{22}+PB_{22}B_{22}^{T}P^{T}\right]
\end{align*}


The three terms inside the brackets are referred as (3.1),(3.2) ,(3.3)
and (3.4), respectively.

Note that $\Tr\left[PP^{T}A_{22}^{T}A_{22}-B_{22}^{T}P^{T}A_{22}P-P^{T}A_{22}^{T}PB_{22}+PB_{22}B_{22}^{T}P^{T}\right]$
can be further simplified to 

\[
\Tr\left[PP^{T}A_{22}^{T}A_{22}-2*P^{T}A_{22}^{T}PB_{22}+PB_{22}B_{22}^{T}P^{T}\right]
\]
.

The gradient for rQAP2 with hard seeds (minimization problem) is

$\nabla_{P}f(P)=-2A_{21}B_{21}^{T}+2PB_{21}B_{21}^{T}-2A_{12}^{T}B_{12}+2A_{12}^{T}A_{12}P+2(A_{22}^{T}A_{22}P+PB_{22}B_{22}^{T}-A_{22}^{T}PB_{22}-A_{22}PB_{22}^{T}$)

\begin{flushleft}
The line search function in terms of $\alpha$ is
\par\end{flushleft}

\begin{flushleft}
\begin{align*}
g(\alpha) & = & \alpha^{2} & \Tr\biggl[\hat{P}^{T}\hat{P}\left(B_{21}B_{21}^{T}+B_{22}B_{22}^{T}\right)+\left(A_{12}^{T}A_{12}+A_{22}^{T}A_{22}\right)\hat{P}\hat{P}^{T} & (1.3+3.4)+(2.1+3.1)\\
 &  &  & -\hat{P}^{T}A_{22}^{T}\hat{P}B_{22}-\hat{P}^{T}A_{22}\hat{P}B_{22}^{T}\biggr] & -(3.2)-(3.3)\\
 & + & \left(1-\alpha\right)^{2} & \Tr\biggl[\hat{Q}^{T}\hat{Q}\left(B_{21}B_{21}^{T}+B_{22}B_{22}^{T}\right)+\left(A_{12}^{T}A_{12}+A_{22}^{T}A_{22}\right)\hat{Q}\hat{Q}^{T} & (1.3+3.4)+(2.1+3.1)\\
 &  &  & -\hat{Q}^{T}A_{22}^{T}\hat{Q}B_{22}-\hat{Q}^{T}A_{22}\hat{Q}B_{22}^{T}\biggr] & -(3.2)-(3.3)\\
 & + & \alpha\left(1-\alpha\right) & \Tr[\left(\hat{Q}^{T}\hat{P}+\hat{P}^{T}\hat{Q}\right)\left(B_{21}B_{21}^{T}+B_{22}B_{22}^{T}\right)+\left(A_{12}^{T}A_{12}+A_{22}^{T}A_{22}\right)\left(\hat{Q}\hat{P}^{T}+\hat{P}\hat{Q}^{T}\right) & (1.3)+(3.4)+(2.1)+(3.1)\\
 &  &  & -\hat{P}^{T}\left[A_{22}^{T}\hat{Q}B_{22}+A_{22}\hat{Q}B_{22}^{T}\right]-\hat{Q}^{T}\left[A_{22}^{T}\hat{P}B_{22}+A_{22}\hat{P}B_{22}^{T}\right]] & -[(3.3)+(3.2)]-[(3.3)+(3.2)]\\
 & + & \alpha & \Tr\left[-2\hat{P}B_{12}^{T}A_{12}-2\hat{P}^{T}A_{21}B_{21}^{T}\right] & [-(2.2)-(1.2)]\\
 & + & \left(1-\alpha\right) & \Tr\left[-2\hat{Q}B_{12}^{T}A_{12}-2\hat{Q}^{T}A_{21}B_{21}^{T}\right] & [-(2.2)-(1.2)]
\end{align*}

\par\end{flushleft}

where the decimal numbers in the right end of the line refer to the
terms for corresponding to $\left\Vert A_{21}-PB_{21}\right\Vert _{F}$
,$\left\Vert A_{12}P-B_{12}\right\Vert _{F}$ and $\left\Vert A_{22}P-PB_{22}\right\Vert _{F}$
in the objective function. Writing $g\left(\alpha\right)$ in terms
of $\alpha$ and (1-$\alpha$),

$g\left(\alpha\right)=c\alpha^{2}+e(1-\alpha)^{2}+d\alpha(1-\alpha)+u\alpha+v(1-\alpha)$

So $c=\Tr\left[\hat{P}^{T}\hat{P}\left(B_{21}B_{21}^{T}+B_{22}B_{22}^{T}\right)+\left(A_{12}^{T}A_{12}+A_{22}^{T}A_{22}\right)\hat{P}\hat{P}^{T}-\hat{P}^{T}A_{22}^{T}\hat{P}B_{22}-\hat{P}^{T}A_{22}\hat{P}B_{22}^{T}\right]$

\noindent 
\begin{eqnarray*}
d & = & \Tr\biggl[\left(\hat{Q}^{T}\hat{P}+\hat{P}^{T}\hat{Q}\right)\left(B_{21}B_{21}^{T}+B_{22}B_{22}^{T}\right)+\left(A_{12}^{T}A_{12}+A_{22}^{T}A_{22}\right)\left(\hat{Q}\hat{P}^{T}+\hat{P}\hat{Q}^{T}\right)\\
 &  & -\hat{P}^{T}\left[A_{22}^{T}\hat{Q}B_{22}+A_{22}\hat{Q}B_{22}^{T}\right]-\hat{Q}^{T}\left[A_{22}^{T}\hat{P}B_{22}+A_{22}\hat{P}B_{22}^{T}\right]\biggr]
\end{eqnarray*}


$e=\Tr\left[\hat{Q}^{T}\hat{Q}\left(B_{21}B_{21}^{T}+B_{22}B_{22}^{T}\right)+\left(A_{12}^{T}A_{12}+A_{22}^{T}A_{22}\right)\hat{Q}\hat{Q}^{T}-\hat{Q}^{T}A_{22}^{T}\hat{Q}B_{22}-\hat{Q}^{T}A_{22}\hat{Q}B_{22}^{T}\right]$

$u=\Tr\left[-2\hat{P}B_{12}^{T}A_{12}-2\hat{P}^{T}A_{21}B_{21}^{T}\right]$

$v=\Tr\left[-2\hat{Q}B_{12}^{T}A_{12}-2\hat{Q}^{T}A_{21}B_{21}^{T}\right]$

Putting this polynomial of $\alpha$ in standard form, we get $a=c+e-d$,
$b=d-2e+u-v$ and $c=e+v$ .
\end{document}
